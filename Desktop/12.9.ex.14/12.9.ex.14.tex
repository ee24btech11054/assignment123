%iffalse
\let\negmedspace\undefined
\let\negthickspace\undefined
\documentclass[journal,12pt,onecolumn]{IEEEtran}
\usepackage{cite}
\usepackage{amsmath,amssymb,amsfonts,amsthm}
\usepackage{algorithmic}
\usepackage{graphicx}
\usepackage{textcomp}
\usepackage{xcolor}
\usepackage{txfonts}
\usepackage{listings}
\usepackage{enumitem}
\usepackage{mathtools}
\usepackage{gensymb}
\usepackage{comment}
\usepackage[breaklinks=true]{hyperref}
\usepackage{tkz-euclide} 
\usepackage{listings}
\usepackage{gvv}                                        
%\def\inputGnumericTable{}                                 
\usepackage[latin1]{inputenc}                                
\usepackage{color}                                            
\usepackage{array}                                            
\usepackage{longtable}                                       
\usepackage{calc}                                             
\usepackage{multirow}                                         
\usepackage{hhline}                                           
\usepackage{ifthen}                                           
\usepackage{lscape}
\usepackage{tabularx}
\usepackage{array}
\usepackage{float}
\usepackage{multicol}
\usepackage{subcaption}

\newtheorem{theorem}{Theorem}[section]
\newtheorem{problem}{Problem}
\newtheorem{proposition}{Proposition}[section]
\newtheorem{lemma}{Lemma}[section]
\newtheorem{corollary}[theorem]{Corollary}
\newtheorem{example}{Example}[section]
\newtheorem{definition}[problem]{Definition}
\newcommand{\BEQA}{\begin{eqnarray}}
\newcommand{\EEQA}{\end{eqnarray}}
\newcommand{\define}{\stackrel{\triangle}{=}}
\theoremstyle{remark}
\newtheorem{rem}{Remark}


% Marks the beginning of the document
\begin{document}
\bibliographystyle{IEEEtran}
\vspace{3cm}

\title{NCERT-12.9.Ex.14}
\author{S. Sai Akshita - EE24BTECH11054}
\newpage
\maketitle
\bigskip

\renewcommand{\thefigure}{\theenumi}
\renewcommand{\thetable}{\theenumi}
\textbf{Question:} In a bank, principal increases continuously at the rate of $5\%$ per year. In how many years Rs 1000 double itself?\\
\textbf{Theoretical Solution:}Let $P$ be the principle at any time $t$. According to the given problem,
\begin{align}
    \frac{dp}{dt}&=\brak{\frac{5}{100}}\times P\\
\frac{dp}{dt}&=\frac{P}{20} \label{eq.1}
\end{align}
separating the variables in \ref{eq.1}, we get
\begin{align}
    \frac{dp}{P}=\frac{dt}{20}\label{eq.2}
\end{align}
Integrating on both sides of \ref{eq.2}, we get
\begin{align}
    \log P&=\frac{t}{20} + C_1\\
    P&=e^{\frac{t}{20}}\times e^{C_1}\\
    P&= Ce^{\frac{t}{20}} \label{eq.3}
\end{align}
Now, when $t=0$, $P=1000$.\\ Substituting the values of $P$ and $t$ in \ref{eq.3}, we get $C=1000$. Therefore, \ref{eq.3} gives
\begin{align}
    P=1000e^{\frac{t}{20}}
\end{align}
Let $t$ years be the time required to double the principal. Then
\begin{align}
    2000&=1000e^{\frac{t}{20}}\\
    t&=20\ln 2 \approx 13.86
\end{align}
\textbf{Solution Using Trapezoid Rule:}
Formula of Trapezoidal rule over an interval $\brak{a,b}$ divided into $n$ equal subintervals is:
\begin{align}
    \int_a^b f\brak{x} \, dx \approx \frac{h}{2} \sbrak{ f\brak{x_0} + 2f\brak{x_1} + 2f\brak{x_2} + \dots + 2f\brak{x_{n-1}} + f\brak{x_n} }
\end{align}
If we do it for one subinterval:
\begin{align}
    \int_{a+nh}^{a+\brak{n+1}h} f\brak{x} \, dx \approx \frac{h}{2}  \sbrak{ f\brak{a+nh} + f\brak{a+\brak{n+1}h} }\label{eq.4}
\end{align}
To obtain difference equation using Trapezoidal rule, we need to integrate \ref{eq.2} for one subinterval.
\begin{align}
    \int_{P\brak{t_n}}^{P\brak{t_{n+1}}} dp &= \int_{t_n}^{t_{n+1}} \frac{Pdt}{20}
\end{align}
from \ref{eq.4}
\begin{align}
     \sbrak { P\brak{t_{n+1}} -P\brak{t_n}}=\frac{h}{2}\sbrak{\frac{P\brak{t_n}+P\brak{t_{n+1}}}{20}}\label{t.r}
\end{align}
Here, $h=t_{n+1}-t_n$, Which gives
\begin{align}
   \sbrak { P\brak{t_{n+1}} -P\brak{t_n}}&=\frac{t_{n+1}-t_n}{2}\sbrak{\frac{P\brak{t_n}+P\brak{t_{n+1}}}{20}} \\ 
   P_{n+1}&=P_n\sbrak{\frac{40+t_{n+1}-t_n}{40-t_{n+1}+t_n}} \label{eq.5} \\
   P_{n+1}&=P_n\sbrak{\frac{40+h}{40-h} }\label{eq.6}
\end{align}
Depending upon the length of subinterval we choose, the accuracy of the result is determined.
By considering one subinterval has $1$ year time gap, and $P_0=1000$, $P_n=2000$, $t_0=0$, and using \ref{eq.5} in recursion, we can find $n$, which indicates the time required to double the principal amount. The smaller the value of $h$, the more accurate the estimated time will be.\\
Choosing $h=1yr$,
\begin{align}
    P_{n+1}&=P_n\sbrak{\frac{41}{39} }\\
    P_{1}&=P_0\sbrak{\frac{41}{39} }\\
    P_{2}&=P_1\sbrak{\frac{41}{39} }\\
    \vdots \\
    P_{n}&=P_0\sbrak{\frac{41}{39} }^{n} \label{eq.7}
\end{align}
Substituting $P_0=1000$ and $P_n=2000$ in \ref{eq.7}, we get $n\approx 13.86$,which means $P_{13}<2000<P_{14}.$
\textbf{Usage of Bilinear transform:}
For the equation \ref{eq.1} take the Laplace of the RHS to be  $X\brak{s}$.
\begin{align}
    \frac{dp}{dt}&=h\brak{t}\\
    h\brak{t}&= \frac{p}{20}
\end{align}
Applying Laplace Transform on both sides and using Laplace properties, we get
\begin{align}
    sY\brak{s}=X\brak{s}
\end{align}
Substituting in the Transfer Function,
\begin{align}
    H\brak{s}&=\frac{Y\brak{s}}{X\brak{s}}\\
    H\brak{s}&=\frac{1}{s}\label{eq.8}
\end{align}
Now, we have to apply Bilinear Transform on both sides of \ref{eq.8}, which is conversion of $s$-domain to $z$-domain.
\begin{align}
    s&=\frac{2}{h}\frac{1-\frac{1}{z}}{1+\frac{1}{z}}\\
    H\brak{z}&=\frac{h}{2}\frac{1+\frac{1}{z}}{1-\frac{1}{z}}\\
    \frac{Y\brak{z}}{X\brak{z}}&=\frac{h}{2}\frac{1+\frac{1}{z}}{1-\frac{1}{z}}\\
    \brak{1-\frac{1}{z}}Y\brak{z}&=\frac{h}{2}\brak{1+\frac{1}{z}}X\brak{z}\label{eq.9}
\end{align}
Apply Inverse z-transform on both sides of \ref{eq.9},
\begin{align}
    P_{n+1}-P_{n}&=\frac{h}{2}\brak{h\brak{t_n}+h\brak{t_{n+1}}}\\
    P_{n+1}&=P_{n}+\frac{h}{2}\brak{h\brak{t_n}+h\brak{t_{n+1}}}\\
    P_{n+1}&=P_{n}+\frac{h}{2}\brak{\frac{P\brak{t_n}+P\brak{t_{n+1}}}{20}}
\end{align}
which is same as \ref{t.r}.


\begin{figure}[h!]
    \centering
    \includegraphics[width=\textwidth]{principal.jpeg} 
\end{figure}

   
\end{document}
